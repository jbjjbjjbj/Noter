\title{Exam Notes}
\tableofcontents

\section{Linear Models}

This section explains a set of methods, which are all summarized in \cref{tab:lin:compare}.

\subsection{K Nearest Neighbour}

Near neighbours tend to have the same label.
Therefore prediction is done with the label that occurs most frequently among the $k$ nearest neighbours to the node itself.
Here distance is in accordance to a $D$-dimensional input feature space.

\begin{mdframed}[frametitle={A Note on Features}]
    \begin{itemize}
        \item[$\mathbf{x}$] Information we get raw, such as \emph{attributes}, \emph{(input) features}, \emph{(predictor) variables}.
        \item[$\mathbf{x'}$] Selected, transformed, or otherwise ``engineered'' features.
        \item[$\mathbf{x''}$] A transformation of $x'$ that is automaticly constructed by the machine learning method.
    \end{itemize}
\end{mdframed}

\begin{description}
    \item[Instance space] Space of all possible values of input features $\mathbf{x}$.
    \item[Decision regions] A classifier devides the instance space into decision regions.
\end{description}

\subsection{Perceptron and Naive Bayes}

Linear models, which are therefore limited in the applicability (they cannot predict XOR function).
However they are usefull baselines, and can be trained well with limited datasets.
They are also integral components in \emph{support vector machines} and \emph{deep neural networks}.

Less unlikely to overfit, however we can still apply techniques to prevent it.

\subsubsection{The Perceptron}

A neural network with a input layer, no hidden layers, and a single output neuron with a \emph{sign} activation function.
This is represented as,
\[
    O(x_1, \dots, x_n) = \left\{
        \begin{array}{ll}
            1 & \mathrm{if}\; w_0 + w_1 \cdot x_1 + \dots + w_n \cdot x_n > 0 \\
            -1 & \mathrm{otherwise}
        \end{array}
        \right.\,.
\]

This can be used to classify a binary class, where the decisions are seperated by a \emph{linear hyperplane}.
Therefore it is not possible to classify something like the XOR function.

\subsubsection{Naive Bayes Model}

Here we instead work with probabilities, as opposed to the perceptron.
It assumes that attributes are independent, given the class label.

Prediction is done by comparing the probability of each label, and then choosing the most likely.
Thus if labels are $\oplus$ and $\oslash$, we choose $\oplus$ if
\[
    P(\oplus \mid X_1, ..., X_n) \geq P(\oslash \mid X_1, ..., X_n)\,.
\]

\subsection{Overfitting}

Our hypothesis overfits the training data if there exists another hypothesis which performs worse in traning but better in testing.
Thus we learn our training data too well, thus not capturing the general characteristics of what we are predicting.

This can be the case when the \emph{hypothesis space} is very large, refering to the complexity of the learned structure.

\subsection{Linear Functions and Decision Regions}

Linear function can be written with scalar values or vectors, as follows:
\[
    y(x_1, ..., x_D) = w_0 + w_1 \cdot w_1 + \dots + w_D \cdot x_D = w_0 + \mathbf{w} \cdot \mathbf{x}\,.
\]
Using it to make decisions, is done with \emph{decision regions},
\begin{align*}
    \mathcal{R}_1 &= \{\mathbf{x} \mid y(\mathbf{x}) \geq 0\} \\
    \mathcal{R}_2 &= \{\mathbf{x} \mid y(\mathbf{x}) < 0 \}
\end{align*}

Look at the geometry, $\mathbf{w}$ is the direction of the decision boundary between the two binaries.
And $w_0 / || \mathbf{w} ||$, is the distance between the decision boundary and origin.

\subsubsection{Multiple Classes}

If more than two classes is wanted, one can use multiple linear functions in combination.
There are different approaches to this, and they have acompanying figures in the slides.
\begin{itemize}
    \item Multiple binary ``one against all'' classification.
    \item Multiple binary ``one against one'' classification.
\end{itemize}

Instead one can construct a \emph{discriminant function} for each of the class labels.
Then we can classify some input, $\mathbf{x}$, as a label if the corrosponding function is maximal.

\subsubsection{Least Squares Regresssion}

For each data case $\mathbf{x}_n$ have a \emph{target vector}, where class labels are encoded with \emph{one-hot encoding}.
Then we try to minimize
\[
    E_D(\mathbf{\tilde{W}}) = \frac 1 2 \sum_{n=1}^N || \mathbf{\tilde{W}}^T \mathbf{\tilde{x}}_n - \mathbf{t}_n ||^2\,.
\]

It should be noted that this method does not minimize classification errors, and outliers may cause a learned function to not seperate linearly seperable data correctly.

\subsection{Probabilistic Models}

Here we classify $\mathbf{x}$ as class $k$ if $P(Y = k \mid \mathbf{x})$ is maximal.
There are two approaches.

\paragraph{Generative Approach} Where we classify from simple assumptions about the distribution of the data.
    Thus we model the probabilities $P(\mathbf{x} \mid k)$ (class-conditional) and $P(k)$ (class prior) and from this calculate $P(k \mid \mathbf{x})$ (posterior).
    Examples of this include \emph{Naive Bayes} and \emph{LDA}.
    Has the consequence that we can generate example data.

    When learning, the goal is to maximize the likelihood:
    \[
        \prod_{n=1}^N P(\mathbf{x}_n, y_n)\,.
    \]

\paragraph{Discriminative Approach} Here we directly learn the distribution $P(Y \mid \mathbf{X})$.
    Examples are \emph{logistic regression} and \emph{neural networks}.

    Likewise the learning goal is to maximize the likelihood:
    \[
        \prod_{n=1}^N P(y_n \mid \mathbf{x}_n)\,.
    \]
    Note that this is a conditional probablity and not a joint probability.

\begin{mdframed}[frametitle={Sigmoid, Logit, and log-odds}]
    The function $\sigma(x)$, also called the \emph{logistic sigmoid}, is defined by
    \[
        \sigma(x) = \frac 1 {1 + e^{-x}}\,.
    \]
    This function maps the whole real axis into a finite interval.

    The \emph{logit} function is defined as the inverse of $\sigma(x)$,
    \[
        x = \ln \left( \frac \sigma {1 - \sigma} \right)\,,
    \]
    and represents the ratio
    \[
        \ln \left( \frac {P(k_1 \mid \mathbf{x})} {P(k_2 \mid \mathbf{x})} \right)
    \]
    for the two binary classes, and is therefore known as the \emph{log-odds}.
\end{mdframed}

\subsubsection{Naive Bayes}

\begin{figure}[H]
    \centering
    \begin{tikzpicture}
        \node[draw, circle] (y) {$\;Y\;$};
        \node[draw, circle, below=of y] (x2) {$X_2$};
        \node[draw, circle, right=of x2] (x3) {$X_3$};
        \node[draw, circle, left=of x2] (x1) {$X_1$};

        \draw[->] (y) edge (x1) (y) edge (x2) (y) edge (x3);
    \end{tikzpicture}
\end{figure}

Here $P(Y \mid \mathbf{x})$ is a probability table for each class.
And $P(X \mid Y, \mathbf{w})$ is if $X_i$ is categorical a table, otherwise multiple gaussian distributions with parameters defined by $\mathbf{w}$.

\subsubsection{Gaussian Mixture Models}

Here we do not have the assumption that input variables are independent as with the naive bayes method.
Instead we model $P(\mathbf{X} \mid Y)$ as Gaussian, such that for all classes $k$:
\[
    P(\mathbf{X} = \mathbf{x} \mid Y = y) = \frac 1 {(2\pi)^{D / 2} | \Sigma_k |^{1 / 2}} \cdot e^{-\frac 1 2 (\mathbf{x} - \mathbf{\mu}_k)^T \Sigma_k^{-1}(\mathbf{x} - \mathbf{\mu}_k)}
\]
defined with the parameters $\mathbf{\mu}_k$ (mean vectors) and $\Sigma_k$ (co-variance matrixes).

\subsubsection{Logistic Regression}

A model with continues input and two output classes: $0$ and $1$.
And yes this is classification and not regression (what).
With a weight vector $\mathbf{w}$ and the sigmoid function, we say that
\[
    P(Y = 1 \mid \mathbf{X} = \mathbf{x}, \mathbf{w}) = \sigma(\mathbf{X} \cdot \mathbf{w})
\]
and
\[
    P(Y = 0 \mid \mathbf{X} = \mathbf{x}, \mathbf{w}) = 1 - P(Y = 1 \mid \mathbf{X} = \mathbf{x}, \mathbf{w})\,.
\]
Thus due to the shape of sigmoid, the dicision boundary is $\mathbf{w} \cdot \mathbf{x} \geq 0$.

If $\mathbf{x}$ is of dimensions $M$, then we have $M$ fittable parameters.
This is a contrast with the Gaussian Mixture Models.

\emph{Categorical attributes} can be optained by using one hot encoding on the input vector.
Therefore having a single values for each of the possible categories.
Multiple outputs can also be optained by choosing a reference class, and then compare the rest according to this reference class with multiple models.

When learning we try to maximize the likelyhood
\[
    \prod_{n:y_n=1} P(Y = 1 \mid \mathbf{X} = \mathbf{x}_n, \mathbf{w}) \cdot \prod_{n:y_n=0} (1 - P(Y = 1 \mid \mathbf{X} = \mathbf{x}_n, \mathbf{w}))\,,
\]
which is often done with numerical methods like Newton-Raphson.

\begin{table}[h]
    \caption{Comparison between different methods.}
    \label{tab:lin:compare}
    \begin{tabularx}{\linewidth}{XXXXX}
        \toprule
                  & Least Squares & Perceptron & LDA & Logistic Regression \\ \midrule
        Criterion & Approximate one-hot class & Minimize error function & Maximize likelyhood & Maximize likelyhood \\
        Multi-class & Yes & No (through extension) & Yes & No (through extension)  \\
        Learning & Solving matrix equation & Iterative optimization & One step optimization & Iterative optimization \\
        Finds seperating regions if possible & not always & yes & not always & not always \\
        Works for non seperable data & yes & poorly & yes & yes \\ \bottomrule
    \end{tabularx}
\end{table}

\subsection{Exam Notes}

The topics for the exam are as follows, the missing topics are noted in fat:
\begin{itemize}
    \item Decision Regions
    \item Overfitting
    \item Least squares regression (corresponding to sklearn LinearRegression in self study 1)
    \item Linear discriminant analysis
    \item Logistic Regression
\end{itemize}

\begin{mdframed}[frametitle={Questions that Need Answering}]
    \begin{itemize}
        \item Løs opgaver for denne lektion.
    \end{itemize}
\end{mdframed}

\subsubsection{Self Study 1}

\begin{enumerate}
    \item Dataset contains for \emph{features}. We show a plot of two of the features, where two of the three classes are \emph{linearly seperable}.
    \item We create 4 models one of which is the linear regression.
    \item We know note the \emph{decision regions}, some important points:
        \begin{itemize}
            \item KNN can have multiple decision regions for the same class, and the \emph{boundaries} are not linear.
                This is clear when we have a $N=1$, and a green outlier has its own small region.
            \item With linear regression, we use 3 \emph{decision functions} and then choose the class which has the largest output.
                In this case, this does not work that well, given that least squares comes from regression which assumes a gaussian distribution which does not fit binary values.
        \end{itemize}
    \item Then we split the data using a seed of one, and predict with KNN ($N=1$), KNN ($N=3$), and linear regression.
        We see that linear regression gives much worse results, which makes sense given the decision regions that where drawn.
        From the \emph{confusion matrix} we se that green and blue are very confusing which makes sense.
    \item We now try with all the features.
        Here we see that KNN ($N=3$) gives the best results on the testing data, however actually performs worse than $N=1$ on the training data.
        This may be due to the \emph{overfitting} which happens when increasing the complexity of the model structure.

        Linear also performs much better, due to it being able to use other features the linearly seperate classes.
\end{enumerate}


\section{Support Vector Machines}

\subsection{Maximum Margin Hyperplanes}

\emph{Maximum-margin hyperplace} is a hyperplane with the maximum amount of margin to all points.
Here distance to closest point in each class is the same, making it more likely that new cases are in the correct regions.
We then now that each class has datapoints, whose distance to the hyperplane equals the margin, and these points are called \emph{support vectors}.


\subsection{Feature Space}

We can create a mapping, $\phi : \mathbb{R}^D \rightarrow \mathbb{R}^{D'}$, which transforms the original data:
\[
    \phi(\mathbf{x}) = (\phi_1(\mathbf{x}), \dots, \phi_{D'}(\mathbf{x}))
\]
Here the components $\phi_i$ are called \emph{features} or \emph{basis functions} and $\mathbb{R}^{D'}$ is the \emph{feature space} of $\phi$.
This mapping can be usefull, to transform data such that it is linearly seperable.

\subsection{Nonlinear SVM}

\subsection{Exam Notes}

\begin{itemize}
    \item Maximum margin hypeplanes
    \item Feature transformations and kernel functions
    \item The kernel trick
    \item String kernels
\end{itemize}

