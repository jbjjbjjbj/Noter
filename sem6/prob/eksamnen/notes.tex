\title{Eksamnens Noter}


The universal set or sample space is the set everything, and is denoted $S$.
Therefore the probability of hitting $S$ is $P(S) = 1$.

This is the first of 3 axioms repeated below.

\begin{enumerate}
    \item For any event $A$, $P(A) \geq 0$.
    \item The probability of hitting sample space is always 1, $P(S) = 1$.
    \item If events $A_1, A_2, ...$ are \textbf{disjoint} event, then
        \begin{equation}
            P(A_1 \cup A_2 ...) = P(A_1) + P(A_2)\,.
        \end{equation}
\end{enumerate}

The last axiom requires that the events $A_n$ are disjoint.
If they aren't one should subtract the part they have in common.
This is called the \emph{Inclusion-Exclusion Principle}.

\begin{principle}
    The \emph{Inclusion-Exclusion Principle} is defined as
    \begin{equation}
        P(A \cup B) = P(A) + P(B) - P(A \cap B)\,.
    \end{equation}
    Definition with 3 events can be found in the in the book.
\end{principle}

\section{Counting}

The probability of a event $A$ can be found by
\begin{equation}
    P(A) = \frac {|A|} {|S|}\,.
\end{equation}
It is therefore required to count how many elements are in $S$ and $A$.
The most simple method is the \emph{multiplication principle}.

\begin{principle}[Multiplication principle]
    Let there be $r$ random experiments, where the $k$'th experiment has $n_k$ outcomes.
    Then there are
    \begin{equation}
        n_1 \cdot n_2 \cdot ... \cdot n_r
    \end{equation}
    possible outcomes over all $r$ experiments.
\end{principle}

