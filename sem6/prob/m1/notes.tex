\title{Eksamnens Noter}


The universal set or sample space is the set everything, and is denoted $S$.
Therefore the probability of hitting $S$ is $P(S) = 1$.

This is the first of 3 axioms repeated below.

\begin{enumerate}
    \item For any event $A$, $P(A) \geq 0$.
    \item The probability of hitting sample space is always 1, $P(S) = 1$.
    \item If events $A_1, A_2, ...$ are \textbf{disjoint} event, then
        \begin{equation}
            P(A_1 \cup A_2 ...) = P(A_1) + P(A_2)\,.
        \end{equation}
\end{enumerate}

The last axiom requires that the events $A_n$ are disjoint.
If they aren't one should subtract the part they have in common.
This is called the \emph{Inclusion-Exclusion Principle}.

\begin{principle}
    The \emph{Inclusion-Exclusion Principle} is defined as
    \begin{equation}
        P(A \cup B) = P(A) + P(B) - P(A \cap B)\,.
    \end{equation}
    Definition with 3 events can be found in the in the book.
\end{principle}

\section{Counting}

The probability of a event $A$ can be found by
\begin{equation}
    P(A) = \frac {|A|} {|S|}\,.
\end{equation}
It is therefore required to count how many elements are in $S$ and $A$.
The most simple method is the \emph{multiplication principle}.

\begin{principle}[Multiplication principle]
    Let there be $r$ random experiments, where the $k$'th experiment has $n_k$ outcomes.
    Then there are
    \begin{equation}
        n_1 \cdot n_2 \cdot ... \cdot n_r
    \end{equation}
    possible outcomes over all $r$ experiments.
\end{principle}

Possible outcomes when choosing $k$ objects from a basket with $n$ objects.
\begin{itemize}
    \item Sampling with Replacement and with Ordering
        \[
            n^k
        \]
    \item Sampling without Replacement and with Ordering
        \[
            n(n - 1) \dots (n - k - 1)
        \]
    \item Sampling without Replacement and without Ordering
        \[
            \frac{n(n-1) \dots (n - k - 1)}{k!} = \frac {n!} {(n- k)! k!} = \binom n k
        \]
    \item Sampling with Replacement and without Ordering
        \[
            \binom {n - 1 - k} k
        \]
\end{itemize}

\section{Independence of Events}

Describes whether the outcome of event $B$, changes the changes of event $A$ occuring.
If this is false, it can be said that $P(A|B) = P(A)$.

\begin{definition}[Independence]
    Two events are independent if
    \[
        P(A \cap B) = P(A) P(B)
    \]
\end{definition}

\section{Mixed Definitions}

\begin{definition}[Conditional Probability]
    The probability that $A$ occurs given that $B$ occurs:
    \[
        P(A|B) = \frac {P(A \cap B)} {P(B)}\,,\:P(B) > 0
    \]
\end{definition}

\begin{definition}[Baye's Rule]
    For any two events $A$ and $B$, where $P(A) \neq 0$, we have that
    \[
        P(B|A) = \frac {P(A|B) P(B)} {P(A)}\,.
    \]
\end{definition}

